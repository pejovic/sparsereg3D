\documentclass[a4paper]{article}

\title{A \emph{brew} Test File}
\author{\emph{Jeffrey Horner}}

\usepackage{a4wide,graphicx}

\begin{document}

\maketitle

A simple example that will run in any \emph{R} engine \emph{(brew may work with S, but this is untested)}: The integers from 1 to
10 are
\begin{verbatim}
 1  2  3  4  5  6  7  8  9 10
\end{verbatim}

We can also emulate a simple calculator \emph{(although the syntax is not as elegant as Sweave)}:
\begin{verbatim}
> 1+1
[1] 2

> 1+pi
[1] 4.141593

> 1+pi
[1] 4.141593

> sin(pi/2)
[1] 1

\end{verbatim}

Now we look at Gaussian data:

\begin{verbatim}
 [1]  0.7267507  1.1519118  0.9921604 -0.4295131  1.2383041 -0.2793463  1.7579031  0.5607461 -0.4527840 -0.8320433 -1.1665705 -1.0655906 -1.5637821
[14]  1.1565370  0.8320471 -0.2273287  0.2661374 -0.3767027  2.4413646 -0.7953391

	One Sample t-test

data:  x
t = 0.81782, df = 19, p-value = 0.4236
alternative hypothesis: true mean is not equal to 0
95 percent confidence interval:
 -0.3067755  0.7002617
sample estimates:
mean of x 
0.1967431 


\end{verbatim}

Note that we can easily integrate some numbers into standard text: The
third element of vector \texttt{x} is 0.9921604, the
$p$-value of the test is 0.4236. % $

Now we look at a summary of the famous iris data set, and we want to
see the commands in the code chunks \emph{(brew can't show you the code, so
don't look for it)}:

\begin{verbatim}
  Sepal.Length    Sepal.Width     Petal.Length    Petal.Width          Species  
 Min.   :4.300   Min.   :2.000   Min.   :1.000   Min.   :0.100   setosa    :50  
 1st Qu.:5.100   1st Qu.:2.800   1st Qu.:1.600   1st Qu.:0.300   versicolor:50  
 Median :5.800   Median :3.000   Median :4.350   Median :1.300   virginica :50  
 Mean   :5.843   Mean   :3.057   Mean   :3.758   Mean   :1.199                  
 3rd Qu.:6.400   3rd Qu.:3.300   3rd Qu.:5.100   3rd Qu.:1.800                  
 Max.   :7.900   Max.   :4.400   Max.   :6.900   Max.   :2.500                  

\end{verbatim}


\begin{figure}[htbp]
  \begin{center}
\includegraphics{brew-test-1-1}
    \caption{Pairs plot of the iris data.}
  \end{center}
\end{figure}


\begin{figure}[htbp]
  \begin{center}
\includegraphics{brew-test-1-2}
    \caption{Boxplot of sepal length grouped by species.}
  \end{center}
\end{figure}

\end{document}


